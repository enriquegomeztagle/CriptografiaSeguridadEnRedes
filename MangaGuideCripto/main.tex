\documentclass{article}
%%%%%%%%%%%%%%%%%%%%%%%%%%%%%%%%%%%%%%%%%%%%%%%%%%%%%%%%%%
\usepackage{graphicx}
\usepackage{multirow}
\usepackage{amsmath,amssymb,amsfonts,amsthm, mathrsfs}
\usepackage[title]{appendix}
\usepackage{xcolor}
\usepackage{textcomp}
\usepackage{manyfoot, booktabs}
\usepackage{algorithm, algorithmicx, algpseudocode}
\usepackage{listings}
\usepackage{biblatex}

\usepackage[spanish]{babel}
%%%%%%%%%%%%%%%%%%%%%%%%%%%%%%%%%%%%%%%%%%%%%%%%%%%%%%%%%%
\raggedbottom

\begin{document}

\begin{titlepage}
    \centering

    \vspace{1cm}
    {\huge\bfseries Manga Guide To Cryptography\par}
    \vspace{2cm}
    
    {\Large\itshape Por:\par}
    \vspace{0.5cm}
    {\Large Enrique Ulises Báez Gómez Tagle\par}
    \vspace{1.5cm}
    
    {\large Trabajo presentado para\par}
    {\large la asignatura Criptografía y Seguridad en Redes\par}
    \vspace{1.5cm}
    
    {\large Universidad Panmericana\par}
    {\large Facultad de Ingeniería\par}
    {\large Ciudad de México, México\par}
    \vspace{1cm}
    {\large \today\par}
    
    \vfill
    
\end{titlepage}
%%%%%%%%%%%%%%%%%%%%%%%%%%%%%%%%%%%%%%%%%%%%%%%%%%%%%%%%%%
\title{Manga Guide To Cryptography}

\author{Enrique Ulises Báez Gómez Tagle}

\maketitle
%%%%%%%%%%%%%%%%%%%%%%%%%%%%%%%%%%%%%%%%%%%%%%%%%%%%%%%%%%
\tableofcontents
\newpage
%%%%%%%%%%%%%%%%%%%%%%%%%%%%%%%%%%%%%%%%%%%%%%%%%%%%%%%%%%
\section{Objetivo}
"The Manga Guide to Cryptography" nos enseña varios de los temas fundamentales de la seguridad de datos, algo crítico en 
el área de la inteligencia de datos (donde nosotros nos desempeñamos actualmente). Este libro toca conceptos como los 
algoritmos de cifrado simétrico o asimétrico, que por ejemplo, son cruciales para proteger la integridad de los datos 
cuando hacemos procesamiento y análisis en entornos de big data.

Otros temas que se tocan son la implementación de funciones hash y técnicas de autenticación, 
que también son indispensables en el pipeline de datos actual, o por ejemplo, los conceptos de PKI también son súper 
relevantes en los entornos de datos distribuidos y para la implementación de data lakes seguros.

Para profesionales de datos como nosotros, es importantísimo comprender estos fundamentos de criptografía,
pues la protección de datos sensibles y el cumplimiento de regulaciones de privacidad son inherentes al ciclo de vida del
análisis de datos. Este libro nos ayudará a comprender estos temas de manera efectiva y a aplicarlos en nuestros proyectos.


\section{Introducción}
The Manga Guide es una serie súper interesante que nos acerca a conceptos técnicos usando el formato manga. En el caso 
de la criptografía, esto es especialmente útil para nosotros como científicos de datos, ya que nos permite entender 
conceptos complejos de seguridad a través de una historia entretenida con imagenes y hasta elementos de misterio.

Aquí, seguimos a Ruka Meguro, al Inspector Jun Meguro y a Rio Yoneda, una reportera. Juntos intentan resolver el misterio
de Ms. Cypher, una "ladrona fantasma". A través de esta trama, vamos aprendiendo los fundamentos de la criptografía que 
aplicamos en nuestro trabajo diario con datos.

Es increíble cómo la criptografía está presente en casi todo lo que hacemos en el análisis de datos, desde la 
protección de bases de datos hasta la seguridad en nuestros pipelines de procesamiento. Este libro nos ayuda a 
entender estos conceptos de forma práctica y amena. Si trabajas con datos, te recomiendo muchísimo leerlo completo, 
ya que explica conceptos fundamentales que necesitamos manejar en nuestro día a día como profesionales de datos.

\section{Desarrollo}
    \subsection{Fundamentos de Encriptación}
        \subsubsection{El Cifrado}
            El cifrado es la técnica que usamos para proteger la información sensible, convirtiendo 
        información legible en código secreto que solo pueden descifrar las personas autorizadas.
        
        \subsubsection{Modelo de Shannon}
            Shannon revolucionó la criptografía cuando estableció que todo proceso de cifrado requiere un texto original, un método de 
        encriptación (E) y una llave (K), siendo la desencriptación el proceso inverso usando un algoritmo de descifrado (D).

        \subsubsection{Métodos Clásicos}
            \begin{itemize}
                \item \textbf{Cifrado César:} Simplemente consiste en avanzar cada letra un número fijo de espacios en el abecedario.
                \item \textbf{Cifrado de Sustitución:} Hay que reemplazar las letras del mensaje usando una guía que nos dice qué letra va por cuál.
                \item \textbf{Cifrado Polialfabético:} Es más complejo porque usa diferentes reglas de sustitución que van cambiando según avanzamos en el texto.
                \item \textbf{Cifrado de Transposición:} Aquí, las letras son las mismas, solo que las mezclamos siguiendo cierta regla establecida.
            \end{itemize}
        
        \subsubsection{Seguridad}
            Para poder saber qué tan seguro puede ser un método de cifrado, hay que contemplar dos cosas principales:
            \begin{itemize}
                \item Cuántas llaves diferentes podemos usar (espacio de claves)
                \item Qué tan resistente es contra alguien que intente analizar patrones en el texto
            \end{itemize}
            Cifrados como el de Vernam son "matemáticamente imposibles" de romper porque usan llaves
            aleatorias tan larga como el mensaje mismo. Pero siendo realistas, esto no es muy práctico.
        
            Por eso en el mundo real usamos cifrados que son "suficientemente seguros", no 100\% seguros
            pero usan mucho tiempo computacional o recursos computacionales que sería caro usarlos.

    \subsection{Algoritmos de Llave Simétrica}
            Literal como si tú y tu amigo tienen la misma llave para un candado. 
            La ventaja de la criptografía simétrica es que es rápida y muy buena al momento de
            manejar muchos datos, aunque mucho ojo a quién le compartimos la llave.
        
        \subsubsection{El ABC Digital: Bits y Operaciones}
            Las computadoras hablan en unos y ceros (bits), y juntan estos en grupos de 8 (bytes). 
            Para cifrar, usan trucos matemáticos como el XOR, así podemos recuperar cualquier parte 
            del proceso si tenemos las otras dos.
        
        \subsubsection{Familias de Cifrado}
            \begin{itemize}
                \item \textbf{Cifrados de Flujo}
                Estos cifran la información poco a poco, como una corriente continua. Usan una 
                "receta secreta" muy larga para generar números que parecen aleatorios. 
                RC4 y SEAL son algunos ejemplos famosos.
                \item \textbf{Cifrados por Bloques}
                Funcionan como un rompecabezas: dividen el mensaje en piezas iguales y las cifran 
                una por una. Pueden trabajar con piezas de diferentes tamaños y tienen distintas 
                formas de hacerlo (como ECB y CBC).
            \end{itemize}
            
    
        \subsubsection{DES}
            DES fue el pionero en la encriptación comercial. Funciona dividiendo el mensaje en 
            bloques y aplicando varias "vueltas" de cifrado. Aunque en su momento fue revolucionario, hoy en 
            día su llave es demasiado pequeña para considerarse segura.
        
        \subsubsection{Los nuevos algoritmos: 3-DES y AES}
            \begin{itemize}
                \item \textbf{3-DES:} Como aplicar DES, pero tres veces seguidas, usando tres llaves 
                diferentes.
                \item \textbf{AES:} Es como un laberinto super complejo que mezcla y revuelve la información 
                de formas muy sofisticadas, y usa una red de sustitución-permutación (SPN).
            \end{itemize}

    \subsection{criptografía asimétrica}
        Es como si fuera un candado mágico que cualquiera puede cerrar, pero solo tú puedes abrir. 
        La criptografía asimétrica: usa una llave pública (para que otros te 
        envíen mensajes) y una privada (para que solo tú los leas).
    
    \subsection{Funciones Unidireccionales}
        Tal cual, un tobogán matemático: fácil deslizarse hacia abajo, imposible subir. 
        Es fácil multiplicar números grandes pero muy difícil encontrar sus factores.

    \subsection{Números Primos y Factorización}
        Los números primos son escenciales en de la criptografía, principalmente para encriptar
        nuestra llave pública, y hay varios algoritmos que basan su seguridad en números compuestos
        grandísimos que son difíciles de factorizar en primos.

    \subsection{Operaciones de Módulo}
        Se usan operaciones de módulo para las 4 operaciones básicas: suma, resta, multiplicación 
        y divisón que igual se usa para encriptar por ejemplo llaves públicas. Y estas operaciones
        solo se pueden hacer a ciertos números con características específicas.

    \subsection{Teoremas de Fermat y Euler}
        \begin{itemize}
            \item El Teorema de Fermat describe el comportamiento de los números primos cuando 
            se realizan operaciones de potenciación modular. 
            \item El Teorema de Euler generaliza este concepto para cualquier número compuesto, introduciendo la función totient que cuenta 
            números coprimos. 
        \end{itemize}
    
    \subsection{Aplicaciones Prácticas de la Encriptación}
        La encriptación va más allá de proteger la confidencialidad de datos, abarcando la 
        integridad, autenticidad y no repudio de la información. Algunas de las que más me llamaron
        la atención por el medio en el que me desempeño actualmente fueron:
        \subsubsection{Funciones Hash}
        "Huellas digitales" única para cada mensaje o los datos. Sirven para:
        - Deduplicación de datos
        - Anonimización de información sensible (como PII)
        - Indexación y búsqueda rápida en grandes conjuntos de datos
        - Verificación de integridad en pipelines de datos
        
        \subsubsection{Encriptación Híbrida}
        Combina la velocidad de la encriptación simétrica para los datos y la seguridad de la 
        asimétrica para el intercambio de claves. La usamos para:
        - Almacenamiento seguro en data lakes
        - Transferencia de grandes volúmenes de datos
        - Protección de modelos de ML en producción
        - Securización de APIs de servicios de datos
        
        \subsubsection{PKI (Infraestructura de Clave Pública)}
        Es todo el sistema que permite que las firmas y certificados digitales funcionen de 
        forma confiable. Se aplica en:
        - Autenticación segura en servicios cloud de ML
        - Certificación de origen de datos para modelos
        - Gestión de accesos a recursos compartidos
        - Verificación de integridad en pipelines de MLOps

\section{Conclusiones}

    A lo largo de esta lectura pude comprender los fundamentos de la criptografía y e identificar 
    su aplicación en el análisis de datos para entenderlos mejor. Hoy en día las técnicas de encriptación 
    son fundamentales para el procesamiento seguro de información. Y con lo aprendido en esta lectura, vemos que 
    es posible balancear la seguridad con el rendimiento requerido en entornos de datos modernos.

    En las funciones hash y la encriptación híbrida encontré nuevas herramientas para 
    proteger y diseñar data lakes y pipelines de datos, y también me gustaría probar la infraestructura PKI para intentar facilitar el 
    despliegue seguro de modelos de machine learning en entornos de producción.

\section{Bibliografía}
\textit{Mitani, M., Sato, S., \& Trend-Pro Co., Ltd. (2018). The manga guide to cryptography. No
Starch Press.}
\end{document}

%%%%%%%%%%%%%%
% Carátula
% Índice
% Objetivo (relacionado a los temas revisados en el libro)
% Introducción (de máximo 3 párrafos)
% Desarrollo (su entendimiento de los temas abordados en cada capítulo)
% Conclusiones
% Bibliografía
